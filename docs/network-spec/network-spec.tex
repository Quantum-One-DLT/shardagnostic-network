\documentclass{report}

\usepackage[dvipsnames,x11names]{xcolor}
\usepackage[utf8]{inputenc}
\usepackage[T1]{fontenc}
\usepackage[british]{babel}
\usepackage{etoolbox}
\newbool{isRelease}
\IfFileExists{.isRelease}{\booltrue{isRelease}}{\boolfalse{isRelease}}
\usepackage[margin=2.5cm]{geometry}
\usepackage{amssymb, latexsym, mathtools}
\usepackage{times}
\usepackage{float}
\usepackage{listings}
\usepackage{natbib}
\usepackage{framed}
\ifbool{isRelease}
    {\usepackage[disable]{todonotes}}
    {\usepackage{todonotes}}

\usepackage{amsmath}
\usepackage{stmaryrd}
\usepackage{colonequals}

\usepackage{tikz}
\usetikzlibrary{automata, positioning, arrows}
\tikzset{
    state/.style={
           rectangle,
           rounded corners,
           draw=black, very thick,
           minimum height=2em,
           inner sep=2pt,
           text centered,
           },
}


\usepackage{microtype}
\usepackage{graphicx}

\usepackage[colorlinks,citecolor=blue,linkcolor=blue,anchorcolor=blue,urlcolor=blue]{hyperref}
\usepackage[capitalise,noabbrev,nameinlink]{cleveref}

\usetikzlibrary{arrows}

\newcommand{\coot}[1]{\textcolor{violet}{\emph{#1}}}
\newcommand{\njd}[1]{\textcolor{purple}{\emph{#1}}}
\newcommand{\avieth}[1]{\textcolor{blue}{\emph{#1}}}
\newcommand{\dcoutts}[1]{\textcolor{orange}{\emph{#1}}}
\addtolength{\marginparwidth}{-0.1\marginparwidth}

\newcommand{\var}[1]{\mathit{#1}}
\newcommand{\type}[1]{\mathsf{#1}}
\newcommand{\powerset}[1]{\mathbb{P}(#1)}
\newcommand{\order}[1]{\mathcal{O}\left(#1\right)}
\newcommand{\restrictdom}{\lhd}
\newcommand{\subtractdom}{\mathbin{\slashed{\restrictdom}}}
\newcommand{\restrictrange}{\rhd}

\ifbool{isRelease}
       {
         \newcommand{\wip}[1]{}
         \newcommand{\hide}[1]{}
       }
       {
         \newcommand{\wip}[1]{\color{magenta}{#1}\color{black}}
         \newcommand{\hide}[1]{}
       }
\newcommand{\haddockref}[2]
           {\href{https://The-Blockchain-Company.github.io/shardagnostic-network/#2}
                 {\emph{haddock documentation}: \texttt{#1}}
           }
\newcommand{\trans}[1]{\texttt{#1}}
\newcommand{\state}[1]{\texttt{#1}}
\newcommand{\msg}[1]{\texttt{#1}}
\newcommand{\StIdle}{\state{StIdle}}
\newcommand{\StBusy}{\state{StBusy}}
\newcommand{\StDone}{\state{StDone}}
\newcommand{\MsgDone}{\msg{MsgDone}}

% TODO: the document is using `\langle` and `\rangle` to denote lists, maybe
% it's better to use Haskell notation, will it be more in sync with other docs
% produced by the formal method team?
\renewcommand{\langle}{[}
\renewcommand{\rangle}{]}

\DeclareMathOperator{\dom}{dom}
\DeclareMathOperator{\range}{range}
\DeclareMathOperator*{\argmin}{arg\,min} % thin space, limits underneath in displays
\DeclareMathOperator*{\minimum}{min}
\DeclareMathOperator*{\maximum}{max}

% ---- Connection Manager things
\lstset{
  xleftmargin=2pt,
  stepnumber=1,
  belowcaptionskip=\bigskipamount,
  captionpos=b,
  escapeinside={*'}{'*},
  language=haskell,
  tabsize=2,
  emphstyle={\bf},
  commentstyle=\it,
  stringstyle=\mdseries\rmfamily,
  showspaces=false,
  keywordstyle=\bfseries\rmfamily,
  columns=flexible,
  basicstyle=\small\sffamily,
  showstringspaces=false,
  morecomment=[l]\%,
}
\lstdefinelanguage{cddl}{
  morekeywords={bool,uint,nint,int,float16,float32,float64,float,bstr,bytes,tstr,text},
  morecomment=[l]{;},
  morestring=[b]",
}
\lstdefinestyle{cddl}{
  numbers=left,
  language=cddl,
  columns=fixed,
}

\definecolor{shadecolor}{rgb}{.6,0.6,0.8}
\definecolor{mygreen}{rgb}{0.109804,0.698039,0.341176}
\definecolor{myblue}{rgb}{0.360784,0.423529,255}
\usetikzlibrary{arrows,calc,matrix,shapes}
\tikzset{every scope/.style={>=triangle 60,thick}}
\exhyphenpenalty 10000

% -------

\raggedbottom

\begin{document}

\title{The Sophie Networking Protocol}
\author{
  rmourey_jr Coutts \\
  {\small \texttt{duncan@well-typed.com}} \\
  {\small \texttt{duncan.coutts@blockchain-company.io}}
\and
  Neil Davies \\
  {\small \texttt{neil.davies@pnsol.com}} \\
  {\small \texttt{neil.davies@blockchain-company.io}}
\and
  Karl Knutsson \\
  {\small \texttt{karl.knutsson@blockchain-company.io}}
\and
  Marc Fontaine \\
  {\small \texttt{marc.fontaine@blockchain-company.io}}
\and
  Armando Santos \\
  {\small \texttt{armando@well-typed.com}}
\and
  Marcin Szamotulski \\
  {\small \texttt{marcin.szamotulski@blockchain-company.io}}
\and
  Alex Vieth \\
  {\small \texttt{alex@well-typed.com}}
}
\date{{\small Version 1.3.0, \today}}

\maketitle

\begin{abstract}
  This document provides technical specification of the implementation of the
  \texttt{shardagnostic-network} component of \texttt{bcc-node}. It provides specification of all
  mini-protocols as well multiplexing and low level wire encoding.  It provides
  necessary information about both node-to-node and node-to-client protocols.

  The primary audience for this document are engineers wishing to build
  clients interacting with a node via node-to-client or node-to-node protocols
  or independent implementations of a node.  Although the original
  implementation of \texttt{shardagnostic-network} is done \texttt{Haskell}, this specification is
  made language agnostic. We may provide some implementation details which are
  \texttt{Haskell} specific.
\end{abstract}

\tableofcontents

\section*{Version history}

\begin{description}
\item[Version 1.0.0 Nov 2019, State machines and wire format for Shardagnostic-Network-1.0.0.]
\item[Version 1.1.0 Apr 2021, Connection Manager for Shardagnostic-Network-Framework.]
\item[Version 1.2.0 Apr 2021, tx-submission version 2, local-state-query, keep-alive mini-protocols]
\item[Version 1.3.0 Jul 2021, Review of the multiplexer documentation]
\end{description}
% \include{intro}
\include{architecture}
\include{mux}
\include{miniprotocols}
\include{connection-manager}
% \include{rest}

\appendix
\chapter{Common CDDL definitions}
\label{cddl-common}
\lstinputlisting[style=cddl]{../../shardagnostic-network/test-cddl/specs/common.cddl}

\bibliographystyle{apalike}
\bibliography{references}

\end{document}
